
\chapter{Umsetzung}

\section{Architektur}
Die Architektur des Projekts umfasst folgende Komponenten:
\begin{enumerate}
    \item \textbf{Virtuelles Netzwerk:} Ein Azure Virtual Network mit Subnetzen für die Anwendung und die Datenbank.
    \item \textbf{Datenbank:} Eine PostgreSQL Flexible Server Instanz, die als PaaS-Dienst bereitgestellt wird.
    \item \textbf{Webanwendung:} Mehrere virtuelle Maschinen (VMs) mit einer Load Balancer-Konfiguration, die eine Next.js-Anwendung hosten.
    \item \textbf{Monitoring:} Eine Monitoring Instanz über die mittels pgAdmin, Prometheus und Graphana die Überwachung der Anwendung erfolgt.
    \item \textbf{Automatisierung:} Terraform wird verwendet, um die Infrastruktur zu erstellen, und Ansible, um die VMs zu konfigurieren und die Anwendung bereitzustellen.
\end{enumerate}

\begin{figure}[H]
    \centering
    \includegraphics[width=0.8\textwidth]{resources/images/Architektur.png}
    \caption{Architektur des Projekts}
\end{figure}

\section{Aufbauablauf}

\subsection{Schritt 1: Infrastruktur mit Terraform}
Zuerst müssen die netzwerke und Server mit Terraform erstellt werden. Hierzu wird ein Azure Resource Group erstellt, in der die Ressourcen gruppiert werden. Anschließend wird ein Virtual Network mit Subnetzen für die Anwendung und die Datenbank erstellt. Danach wird eine PostgreSQL Flexible Server Instanz mit privatem Netzwerkzugriff bereitgestellt. Daruf werden zwei VMs für die Webanwendung erstellt und ein Load Balancer konfiguriert. Abchsließend wird ein Monitoring Server erstellt.

\subsection{Schritt 2: Konfiguration mit Ansible}
Konfiguration der App VMs
\begin{itemize}
    \item Installation von Node.js und PostgreSQL-Client auf den VMs.
    \item Bereitstellung der Next.js-Anwendung aus einem Git-Repository.
    \item Konfiguration der Umgebungsvariablen für die Datenbankverbindung.
    \item Erstellung eines Systemd-Dienstes, um die Anwendung automatisch zu starten.
\end{itemize}

Konfiguration des Monitoring Servers
\begin{itemize}
    \item Installation von pgAdmin, Prometheus und Grafana auf dem Monitoring Server.
    \item Konfiguration von pgAdmin zur Verwaltung der Datenbank.
    \item Konfiguration von Prometheus und Grafana zur Überwachung der Anwendung.
    \item Einrichtung von Dashboards in Grafana zur Visualisierung der Metriken.
\end{itemize}
