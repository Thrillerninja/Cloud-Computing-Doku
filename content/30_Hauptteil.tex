
\chapter{Fazit}
\section{Wann ist die Lösung sinnvoll?}
Die gewählte Lösung ist besonders geeignet für:
\begin{itemize}
    \item Anwendungen, die eine hohe Skalierbarkeit und Verfügbarkeit erfordern.
    \item Projekte, die von der Automatisierung der Infrastruktur profitieren.
    \item Teams, die eine konsistente und wiederholbare Bereitstellung benötigen.
\end{itemize}

\section{Was ist der Aufwand?}
Der Aufwand für die Implementierung der Lösung umfasst:
\begin{itemize}
    \item \textbf{Initiale Einrichtung:} Die Erstellung der Terraform- und Ansible-Skripte erfordert Zeit und Fachwissen.
    \item \textbf{Wartung:} Änderungen an der Infrastruktur oder der Anwendung können einfach durch Anpassung der Skripte vorgenommen werden.
    \item \textbf{Kosten:} Die Nutzung von Cloud-Diensten ist abhängig von der Ressourcennutzung und kann bei falscher Planung teuer werden.
\end{itemize}

\section{Vorteile der Lösung}
\begin{itemize}
    \item Automatisierung reduziert menschliche Fehler.
    \item Skalierbarkeit und Flexibilität der Cloud-Dienste.
    \item Sicherheit durch private Netzwerke und SSL-Verbindungen.
\end{itemize}

\section{Nachteile der Lösung}
\begin{itemize}
    \item Abhängigkeit von Cloud-Anbietern.
    \item Komplexität bei der initialen Einrichtung.
    \item Laufende Kosten für die Cloud-Dienste.
\end{itemize}

\chapter{Zusammenfassung}
Das Projekt zeigt, wie moderne Cloud-Technologien wie Terraform und Ansible genutzt werden können, um eine skalierbare und sichere Infrastruktur für eine Webanwendung bereitzustellen. Die Kombination aus IaaS und PaaS ermöglicht eine effiziente Nutzung der Cloud-Ressourcen, während die Automatisierung die Verwaltung vereinfacht. Die Lösung ist ideal für Anwendungen, die hohe Verfügbarkeit und Flexibilität erfordern, und bietet eine solide Grundlage für zukünftige Erweiterungen.