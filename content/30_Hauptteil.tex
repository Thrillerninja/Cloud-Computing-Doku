\chapter{Fazit}

Die im Rahmen dieses Projekts entwickelte Cloud-Infrastrukturlösung demonstriert die modenen Möglichkeiten um Software effektiv mittels DevOps-Praktiken in die Cloud zu briggen und in dieser zu managen. Die Implementierung einer vollständig automatisierten Bereitstellungspipeline mittels Terraform und Ansible schafft eine robuste Grundlage für skalierbare Webanwendungen.

\section{Anwendungsszenarien und Eignung}

Das Projekt hat gezeigt das die Lösung besonders wertvoll für mittelgroße bis große Anwendungen ist, die folgende Anforderungen erfüllen:

\begin{itemize}
    \item Dynamisch skalierbare Infrastruktur zur Bewältigung schwankender Nutzerlast
    \item Hohe Verfügbarkeit durch lastverteilte Redundanz der Anwendungsserver
    \item Nachhaltige Entwicklungsprozesse mit wiederholbaren Bereitstellungsschritten
    \item Professionelles Monitoring für proaktive Problemerkennung
\end{itemize}

Vor allem für Teams, die agil arbeiten und häufige Deployments durchführen, bietet dieser Ansatz erhebliche Vorteile durch die Standardisierung der Infrastruktur als Code.

\section{Aufwand und Ressourcenbedarf}

Die Implementierung der Lösung erfordert eine differenzierte Betrachtung des Aufwands:

\begin{itemize}
    \item \textbf{Initiale Einrichtung:} Der Aufbau der Terraform- und Ansible-Konfigurationen verlangt Fachwissen und eine präzise Planung der Infrastrukturkomponenten.
    \item \textbf{Wartung und Weiterentwicklung:} Nach der initialen Einrichtung reduziert sich der Wartungsaufwand erheblich. Infrastrukturänderungen können durch Anpassungen der Konfigurationsdateien versioniert und getestet werden.
    \item \textbf{Betriebskosten:} Die Cloud-Ressourcen verursachen laufende Kosten, die jedoch optimiert werden können.
\end{itemize}

\section{Stärken der Architektur}

Die gewählte Architektur zeichnet sich durch mehrere Stärken aus:

\begin{itemize}
    \item \textbf{Hohe Ausfallsicherheit} durch redundante Anwendungsserver und Load Balancing
    \item \textbf{Sicherheit durch Design} mit privaten Subnetzen und klar definierten Netzwerkgrenzen
    \item \textbf{Konsistente Umgebungen} durch Infrastructure as Code, die Umgebungsdrift verhindert
    \item \textbf{Umfassende Überwachbarkeit} durch integrierte Monitoring-Lösungen mit Prometheus und Grafana
    \item \textbf{Wartbarkeit} durch klare Trennung von Infrastruktur und Konfiguration
\end{itemize}

\section{Herausforderungen und Einschränkungen}

Trotz der Vorteile bestehen einige Herausforderungen:

\begin{itemize}
    \item \textbf{Cloud-Anbieter-Bindung:} Die Lösung nutzt spezifische Azure-Dienste, was eine potenzielle Abhängigkeit schafft. Eine Multi-Cloud-Strategie würde zusätzlichen Aufwand erfordern. Dies ist mit PaaS Services nur schwer erreichbar.
    \item \textbf{Lernkurve:} Teams ohne DevOps-Erfahrung müssen sich in Tools wie Terraform und Ansible einarbeiten was einen erheblichen Schulungsaufwand bedeutet.
    \item \textbf{Kostenmanagement:} Die flexible Skalierung erfordert proaktives Kostenmonitoring, um Ressourcenverschwendung zu vermeiden.
    \item \textbf{Komplexität:} Die verteilte Architektur erhöht die Komplexität bei der Fehlerdiagnose, was ein umfassendes Monitoring unerlässlich macht.
\end{itemize}

\section{Zusammenfassung}

Das Projekt demonstriert erfolgreich den Einsatz moderner DevOps-Methoden zur Bereitstellung einer skalierbaren Web-Infrastruktur in der Azure-Cloud. Durch die strategische Kombination von Infrastructure as a Service (IaaS) für Anwendungsserver und Platform as a Service (PaaS) für die Datenbank wurde eine ausgewogene Balance zwischen Kontrolle und Verwaltungsaufwand erreicht.

Die vollständige Automatisierung des Bereitstellungsprozesses durch Terraform und Ansible eliminiert manuelle Konfigurationsschritte und stellt sicher, dass die Infrastruktur jederzeit reproduzierbar bleibt. Das integrierte Monitoring-System ermöglicht eine kontinuierliche Überwachung aller Komponenten und schafft die Grundlage für proaktives Ressourcenmanagement.

Diese Architektur bietet einen zukunftssicheren Ansatz für moderne Webanwendungen, der mit den Anforderungen wachsen kann und gleichzeitig einen strukturierten Rahmen für kontinuierliche Verbesserungen der Infrastruktur bietet.
